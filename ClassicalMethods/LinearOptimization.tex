\documentclass[
    NAME={Dr. Helga Ingimundardóttir},
    EMAIL={helgaingim@hi.is},
    FACULTY={Industrial Engineering},
    TITLE={Linear Optimization},
    SUBTITLE={Fundamentals and Applications},
    SEMINAR={VÉL113F},
    DATE={Design and Optimization}
]{HI-latex/hi-beamer}

\begin{document}

    \begin{frame}[allowframebreaks]{Operations Research Overview}
        \begin{itemize}
            \item Operations Research (OR) aims to determine the most efficient way to perform tasks within companies or
            institutions.
            \item It often involves decision-making where complex issues are presented as optimization problems.
            \item The solution to these problems usually lies in maximizing or minimizing a specific function, known as
            the \emph{objective function}.
            \item OR is an interdisciplinary field combining elements of applied mathematics, statistics, probability
            theory, computer science, and more.
            \item A primary focus within OR is on model creation and optimization.
            \item Its historical roots trace back to World War II, emphasizing its significance in practical applications,
            especially in resource allocation.
        \end{itemize}

        \framebreak

        \begin{block}{Process}
            \begin{enumerate}
                \item Problem definition and data collection.
                \item Develop a mathematical model that captures the essence of the subject.
                \item Develop a computer program to work with the model.
                \item Test (verify) the model. Improve the model if necessary.
                \item Implement the model -- usually in the form of a program.
            \end{enumerate}
        \end{block}

        \framebreak

        \begin{block}{Mathematical Model}
            Mimics the most crucial aspects of the task.
            \begin{enumerate}
                \item What decisions need to be made?
                \item What are the options?
                \item What is the outcome (the gain)?
                \item What are the conditions for good decision making?
                \item Which factors influence the decision?
                \item How can we ensure that we've made the right decision?
            \end{enumerate}
        \end{block}

        \begin{block}{Linear Optimization / Linear Programming}
            It's a technique used to find the best possible outcome in a model where the conditions are expressed through
            \emph{linear} equations and inequalities.
        \end{block}

    \end{frame}

    \begin{frame}[allowframebreaks]{Beer Substitute Optimization Formulation}
        \label{example:beer:model}
        \begin{example}[Old Exam Question]
            Before beer was legalized in Iceland, a beer substitute was in demand. Now, imagine a brewery's challenge to recreate
            it. The mix must contain 3-5\% malt, a minimum of 2\% brandy, no more than 7\% vodka, and a hard liquor cap of 10\% to
            avoid an overpowering spirit taste.

            \begin{table}[h]
                \centering
                \scriptsize
                \begin{tabular}{|l|c|c|}
                    \hline
                    \textbf{Ingredient} & \textbf{Alcohol \%} & \textbf{Cost (kr. per liter)} \\
                    \hline
                    Pilsner             & 2.25\%              & 100                           \\
                    Vodka               & 40\%                & 2000                          \\
                    Brandy              & 40\%                & 3000                          \\
                    Malt Extract        & 1.5\%               & 120                           \\
                    \hline
                \end{tabular}
            \end{table}

            \begin{enumerate}
                \item Develop a linear optimization for the strongest (tasty) beer substitute.
                \item Design one for a cost-effective 4\% beer substitute.
            \end{enumerate}
        \end{example}
    \end{frame}


    \begin{frame}{Mathematical Optimization Model}
        \frametitle{Key Components}
        \begin{itemize}
            \item Decision Variables
            \item Objective Function
            \item Constraints
        \end{itemize}
        \bigskip
        Extensive data collection is needed to estimate the model's parameters.
        \emph{Sensitivity analysis} evaluates the effects of changes in individual parameters.
        If the model is significantly sensitive to certain parameters, extra care is needed in their estimation.
        \emph{Stochastic programming} formally handles uncertainties in the parameters.
    \end{frame}

    \begin{frame}{Types of Models}
        \begin{itemize}
            \item Decision variables can be continuous, discrete, or both.
            \item Objective function can have one or more min/max values.
            \item Constraints can be linear or nonlinear.
        \end{itemize}
        \bigskip
        For now we will focus on models with continuous variables, linear objectives, and linear constraints (i.e. \emph{linear
        optimization}). We will also look at so-called integer optimization (decision variables take values \(0,1,2,\ldots\)).
    \end{frame}

    \begin{frame}[allowframebreaks]{Classification of Optimization Problems}
        \begin{itemize}
            \item Continuous Decision Variables
            \begin{itemize}
                \item Unconstrained
                \begin{itemize}
                    \item Nonlinear equations
                    \item Least squares method
                    \item Global optimization
                    \item Non-differentiable optimization
                \end{itemize}
                \item Constrained
                \begin{itemize}
                    \item \emph{Linear Programming}
                    \item Semidefinite programming
                    \item Nonlinearly constrained
                    \item Bound constraints
                    \item Network Optimization
                \end{itemize}
            \end{itemize}
            \item Discrete Decision Variables
            \begin{itemize}
                \item \emph{Integer Programming}
                \item Stochastic programming
            \end{itemize}
        \end{itemize}
    \end{frame}

    \begin{frame}[allowframebreaks]{Solving LP: The Simplex Method}
        Introduction:
        \begin{itemize}
            \item Developed by George Dantzig in 1947.
            \item A popular algorithm for numerical solution of linear programming problems.
            \item Works by moving along edges of the feasible region defined by the constraints to find the optimal solution.
            \item Utilizes the fact that the optimal solution lies at a vertex of the feasible region.
            \item Efficient for large-scale linear problems.
        \end{itemize}
        \framebreak
        Advantages:
        \begin{itemize}
            \item Highly systematic and provides a lot of information about the problem structure.
            \item Can handle additional constraints easily.
        \end{itemize}
        Challenges:
        \begin{itemize}
            \item May face issues like cycling (though rare with anti-cycling rules).
            \item May not be as efficient for some nonlinear problems.
        \end{itemize}

        \begin{block}{Resources}
            \begin{itemize}
                \item For guidance on using the Simplex method, refer to the video \href{https://www.youtube.com/watch?v=TVbLWxN8q7I}{How to use the simplex method} by StudyForce.
                \item For the algorithm details, check \href{https://en.wikipedia.org/wiki/Simplex_algorithm}{Wikipedia's page on the Simplex Algorithm}.
            \end{itemize}
        \end{block}

        \framebreak

        \begin{block}{Layman's Explanation}
            \note{Imagine you're a drone flying above a landscape. This landscape is like a big piece of paper with
            several lines drawn on it, representing the constraints of a problem. Where the lines intersect, they
            form various polygons - triangles, rectangles, and so on. Your job as the drone is to find the highest
            point (or the lowest, depending on the problem) on this landscape, but you can only land on the edges of
            these polygons.

            However, you have a trick up your sleeve! Instead of searching aimlessly, you always start at one corner
            of a polygon (a vertex). From there, you 'walk' along the edges, moving from vertex to vertex, always
            choosing the next edge that takes you higher (or lower). It's as if you're hiking and always taking the
            path that looks like it leads to a higher peak.

            The genius of the Simplex algorithm is that it knows, due to the structure of these problems, that the
            highest (or lowest) point will always be at a corner, or vertex. So, it doesn't waste time checking the
            middle of the edges or the areas inside the polygon.

            As you continue moving from vertex to vertex, there might come a time when all paths around you would
            take you downwards. When that happens, you know you're at the highest peak in your vicinity. And because
            of the nature of the problems the Simplex algorithm solves, this means you're at the highest point
            overall!

            So, in essence, the Simplex algorithm is like a very efficient hiker or drone that always knows the best
            path to take to reach the highest (or lowest) point on a constrained landscape. It never wastes time on
            pointless paths and always goes straight for the goal.
            }

            Imagine you're a drone over a land with marked paths. Your task is to find the highest point. Instead of
            searching randomly:

            \begin{itemize}
                \item You start at a corner (vertex) of a path.
                \item You 'fly' from corner to corner, always choosing the path leading upwards.
                \item You know the highest point is at a corner due to the problem's structure.
                \item Once all paths around you lead downwards, you've found the highest spot!
            \end{itemize}

            In essence, Simplex is like a savvy drone finding the best peak on a constrained landscape.
        \end{block}
    \end{frame}

    \begin{frame}[allowframebreaks]{Beer Substitute Optimization with Simplex}
        \label{example:beer:simplex}
        \begin{example}[Continued from \hyperlink{example:beer:model}{Optimization Formulation}]
            Now, use the Simplex method to solve:
            \begin{enumerate}
                \item Find the linear optimization solution for the strongest (yet tasty) beer substitute.
                \item Determine the solution for a cost-effective 4\% beer substitute.
            \end{enumerate}
        \end{example}

        \begin{alertblock}{Note}
            Remember to set up the problem with the appropriate decision variables, constraints, and objective function
            before applying the Simplex algorithm.
        \end{alertblock}

    \end{frame}

    \begin{frame}{Solving LP: A Computational Approach}
        \begin{itemize}
            \item What is \emph{Gurobi}?
            \begin{itemize}
                \item Gurobi is a state-of-the-art optimization solver.
                \item It efficiently handles linear, quadratic, and mixed-integer optimization problems.
            \end{itemize}

            \item \emph{Language Integration:}
            \begin{itemize}
                \item Supports various languages, notably C++ and Python.
                \item For our purposes, we'll utilize the \emph{Python} implementation (pip package \texttt{gurobipy}).
            \end{itemize}

            \item \emph{Licensing:}
            \begin{itemize}
                \item Gurobi offers a free academic license.
                \item \alert{Important:} To register for the license, ensure you are connected to the university's network.
            \end{itemize}

            \item \emph{Resources:}
            \begin{itemize}
                \item Gurobi provides extensive resources for learners.
                \item Tutorials range from \href{https://www.gurobi.com/jupyter_models/}{introductory to advanced models}.
                \item All tutorials are hosted in Jupyter Notebooks, facilitating hands-on learning.
            \end{itemize}

        \end{itemize}

    \end{frame}

    \begin{frame}[allowframebreaks]{Beer Substitute Optimization with Gurobi}
        \label{example:beer:guriobi}
        \begin{example}[Continued from \hyperlink{example:beer:model}{Optimization Formulation}]
            Now, use the Gurobi method to solve:
            \begin{enumerate}
                \item Find the linear optimization solution for the strongest (yet tasty) beer substitute.
                \item Determine the solution for a cost-effective 4\% beer substitute.
            \end{enumerate}
        \end{example}

        \begin{alertblock}{Note}
            Compare the results from the Simplex method and Gurobi. Are they the same? If not, why?
        \end{alertblock}

    \end{frame}

\end{document}
